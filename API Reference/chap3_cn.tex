%----------------------------------------------------------------------------------------
%	CHAPTER 3
%----------------------------------------------------------------------------------------
\chapter{API Reference}

%\section{Driver}\index{Driver}
%
%\section{API}\index{API}


\subsection{open(index)}\index{open}
\subsection{write(msg)}\index{write}
\subsection{read(bytenum)}\index{read}
This function  
\subsection{asg\_create\_input\_buf(char **flags, double *time\_length, unsigned long cmd\_num,
 \newline unsigned char **buf , int *buf\_length)}\index{asg\_create\_input\_buf}
User don't use this function!
\subsection{asg\_program\_all(char **flags, double *time\_length, unsigned long cmd\_num, 
\newline unsigned char **buf , int *buf\_length)}\index{asg\_program\_all}
This function writes the pulse data to the board and tells the board to start programming the device.For the device,the method of programming follows the following form:
\newline
    \begin{table}[H]
    \newcommand{\tabincell}[2]{\begin{tabular}{@{}#1@{}}#2\end{tabular}}
        \centering
        \begin{tabular}{|c|c|}
            \hline
            \textbf{Parameters}&\textbf{Description}\\
            \hline
            char **flags&\tabincell{l}{The "flags" is a two-dimensional char array,the first dimension represents the length \\of pulses data,and the second dimension is a string such as "10100101",which \\represents the 8 output channels.}\\
            \hline
            double *time\_length&\tabincell{l}{The "time\_length" is a double array,which represents the time length of the pulse\\condition in nanoseconds.The length of the array must be equal to the length of \\"flags" array.}\\
            \hline
            unsigned long cmd\_num&\tabincell{l}{The cmd\_num is an integer that represents the number of pulse condition.}\\
            \hline
            unsigned char **buf&\tabincell{l}{The "buf" is a two-dimensional char array.The length of first dimension is 8,the \\length of second dimension is 10 times of "cmd\_num".}\\
            \hline
            int *buf\_length&\tabincell{l}{The "buf\_length" is an integer array as \{0,0,0,0,0,0,0,0\}.}\\
            \hline
        \end{tabular}
    \end{table}
\subsection{asg\_start\_programming()}\index{asg\_start\_programming}
\subsection{asg\_program\_one()}\index{asg\_program\_one}
\subsection{asg\_stop\_programming()}\index{asg\_stop\_programming}

\subsection{asg\_counter\_create\_input\_buf(char **flags, double *time\_length, unsigned long cmd\_num, 
\newline unsigned char **buf , int *buf\_length)}\index{asg\_counter\_create\_input\_buf}
User don't use this function!
\subsection{asg\_counter\_program\_all(char **flags, double *time\_length,unsigned long cmd\_num,
unsigned char **buf , int *buf\_length)}\index{asg\_counter\_program\_all}
This function writes the count data to the board and tells the board to start programming the device.For the device,the method of programming follows the following form:
\newline
    \begin{table}[H]
    \newcommand{\tabincell}[2]{\begin{tabular}{@{}#1@{}}#2\end{tabular}}
        \centering
        \begin{tabular}{|c|c|}
            \hline
            \textbf{Parameters}&\textbf{Description}\\
            \hline
            char **flags&\tabincell{l}{The "flags" is a two-dimensional char array,the first dimension represents the \\length of count data,and the second dimension is a string such as "1",which \\represents the count channel.}\\
            \hline
             int *time\_length&\tabincell{l}{The "time\_length" is an integer array,which represents the length of the count \\ condition in nanoseconds.The length of the array must be equal to the length \\of "flags" array.}\\
            \hline
            unsigned long cmd\_num&\tabincell{l}{The cmd\_num is an integer that represents the number of count condition.}\\
            \hline
            unsigned char **buf&\tabincell{l}{The "buf" is a two-dimensional char array.The length of first dimension is 1,\\the length of second dimension is 10 times of "cmd\_num".}\\
            \hline
            int *buf\_length&\tabincell{l}{The "buf\_length" is an integer array as \{0\}.}\\
            \hline
        \end{tabular}
    \end{table}
\subsection{asg\_counter\_start\_programming()}\index{asg\_counter\_start\_programming}
\subsection{asg\_counter\_program\_one()}\index{asg\_counter\_program\_one}
\subsection{asg\_counter\_stop\_programming()}\index{asg\_counter\_stop\_programming}


\subsection{asg\_start()}\index{asg\_start}
This function sends a trigger to the board.This trigger will start execution of a pulse program as well as the count program.
\subsection{asg\_stop()}\index{asg\_stop}
This function stops output of board.This function also make the board back to idle condition so that the board can be run again using \textbf{asg\_start()}.
